\documentclass[12pt]{article}


\pagestyle{headings} \thispagestyle{empty}
%\pagestyle{empty}
  \textwidth      6.4in
      \oddsidemargin 0.0in
      \topmargin     -0.4in
      \topskip          0pt
      \headheight      12pt
      \footskip        18pt
%      \footheight      12pt
      \textheight     650pt

\parindent=0cm
\baselineskip=2cm



\renewcommand{\baselinestretch}{1.5}

\begin{document}

\textbf{Name: Taylor Heilman}    \hspace{4in} \textbf{Due: 11-12-15}
\begin{center} \textbf{Math 210: Homework 8 - Fall 2015} \end{center}

\begin{enumerate}

\item Problem 9.12(c):

{PF

Let $x \in f(C) - f(D)$

Therefore $x \in f(C)$ and $x \notin f(D)$

So $x = f(y)$, where $y \in C$ and $x \notin f(D)$

Then if $y \in D, x = f(y)$ then $x \in f(D)$  this is a contradiction because we know $x \notin f(D)$

So $y \notin D$

Hence $x = f(y)$, where $y \in C$ and $y \notin D$

Therefore $x = f(y)$ where $y \in C - D$

So $x \in f(C-D)$

Thus $x \in f(C) -f(D)$

= $x \in f(C-D)$ 

So $f(C) - f(D) \subseteq f(C-D)$


//
}

\item Problem 9.12(f):

{PF

(WTS: $f^{-1}(E-F) \subseteq f^{-1}(E)-f^{-1}(F)$ and $f^{-1}(E) - f^{-1}(F) \subseteq f^{-1}(E-F)$)

$\Rightarrow$

Let $x \in f^{-1}(E-F)$

So $x = f^{-1}(y)  \exists y \in E-F$ such that

$y \in E, Y \notin F$

$\Rightarrow$ $x = f^{-1}(y)$ where $y \in E$ , $y \notin F$

$\Rightarrow$ $x = f^{-1}(y)$ where $y \in E$ , $x =  f^{-1}(y)$ where $y \notin F$

$\Rightarrow$ $x = f^{-1}(E)$ , $x \notin  f^{-1}(F)$

$\Rightarrow$ $x = f^{-1}(E) - f^{-1}(E)$

So  $x \in f^{-1}(E-F) $ 

$\Rightarrow$  $x \in f^{-1}(E) - x \in f^{-1}(F) $ for every x

Hence $f^{-1}(E-F) \subseteq f^{-1}(E)-f^{-1}(F)$
 
$\Leftarrow$

Let $x \in f^{-1}(E)-f^{-1}(F)$

So $\exists $ and $\notin y \in E$ and $ y \notin F$  such that $x =  f^{-1}(y)$

$\Rightarrow$ $x = f^{-1}(y)$  where $y \in E - F$

$\Rightarrow$ $x \in f^{-1}(E-F)$

Hence $x \in f^{-1}(E)-f^{-1}(F) $

$\subseteq  f^{-1}(E-F)$ 

Therefore  $ f^{-1}(E-F) = f^{-1}(E)-f^{-1}(F) $

//
}


\item Problem 9.40:

{PF

Let $x \in A$

Then  $(f  $ o $i_A)(a) = f(i_A(a))$

= $f(a)$

Similarly  $(i_B$ o $f)(a) = i_B(f(a))$

= $f(a)$

Therefore $(f  $ o $i_A) = f $ and $(i_B$ o $f) = f$

//
}

\item Problem 9.46: 

{PF

(a.)

$(g$ o $f)(3,8) = g(f(3,8))$

= $g(3(3)-8, 3+8)$

= $g(1,11)$

=$(1-11, 2(1) + 11)$

=$(-10, 13)$

Hence $(g$ o $f)(3,8) = (-10,13)$

(b.)

Let $(w,x),(y,z) \in A \times B$

Assume $(g$ o $f)(w,x) = (g$ o $f)(y,z)$

So $g(f(w,x)) = g(f(y,z))$

Then $g(3w-x,w+x) = g(3y - z, y+z)$

Thus $(3w - x - w -x, 6w -2x + w +x) = (3y - z - y - z, 6y -2z + y +z)$

So $(2w - 2x, 5w -x) = (2y - 2z, 5y -z)$

Hence $2w - 2x = 2y - 2z$ and $5w - x = 5y -z$

Therefore $w-y = x-z$ and $5(w-y) = x - z$

So $w-y = 5(w-y)$

$4(w-y) = 0$

$w = y$

Plugging this into $w-y = x-z$ and $5(w-y) = x - z$


$0 = x-z$

$x = z$

Hence $(w,x) = (y,z)$

So the function is one to one


(c.)

Let $(x,y) \in B x A$

So $x,y$ are odd and even integers

Hence $ a = \frac{2y - x}{12} \in A$ and $ b = \frac{2y - 7x}{12} \in B$

Therefore $g($ o $f)(a,b) = g(f(a,b))$

= $g(f(\frac{2y - x}{12} , \frac{2y-7x}{12} ))$

= $g(3(\frac{2y-x}{12}) - \frac{2y - 7x}{12}, \frac{2y-x}{12} + \frac{2y - 7x}{12})$

= $g( \frac{4y + 4x}{12}, \frac{4y - 8x}{12})$

= $g( \frac{y + x}{3}, \frac{y -2x}{3})$

= $( \frac{y+x}{3} - \frac{y - 2x}{3}, 2( \frac{y+x}{3} )+ \frac{y-2x}{3})$

= $(x,y)$

So the function is onto

//
}


\item Problem 9.58:

{PF

Let $g: B \rightarrow A$ be surjective

Then $g(B) = A$

So $g(y) = x$

Consider $(f$ o $g)(y) = f(g(y))$

= $f(x)$

= $y$

= $I_B (y)$

Also  $(g$ o $f)(x) = g(f(x))$

= $g(y)$

= $x$

= $I_A (x)$

Therefore $(g$ o $f)(x) = I_A(x))$



//
}


\end{enumerate}

\end{document}






















