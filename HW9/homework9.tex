\documentclass[12pt]{article}
\usepackage{amsmath}


\pagestyle{headings} \thispagestyle{empty}
%\pagestyle{empty}
  \textwidth      6.4in
      \oddsidemargin 0.0in
      \topmargin     -0.4in
      \topskip          0pt
      \headheight      12pt
      \footskip        18pt
%      \footheight      12pt
      \textheight     650pt

\parindent=0cm
\baselineskip=2cm

%\include these lines if you want to use the LaTeX "theorem" environments
\newtheorem{theorem}{Theorem}[section]
\newtheorem{definition}[theorem]{Definition}
\newtheorem{lemma}[theorem]{Lemma}
\newtheorem{corollary}[theorem]{Corollary}
\newtheorem{guess}{Conjecture}
\newtheorem{example}[theorem]{Example}

%include lines like this if you want to define your own commands
%to save typing
\newcommand{\PROOF}{\noindent {\bf Proof}: }
\newcommand{\REF}[1]{[\ref{#1}]}
\newcommand{\Ref}[1]{(\ref{#1})}
\newcommand{\dt}{\mbox{\rm   dt}}
\newcommand{\qed}{\Large{\bf{$\diamond$}}}
\newcommand{\phat}{\hat{p}}

\DeclareSymbolFont{AMSb}{U}{msb}{m}{n}
\DeclareMathSymbol{\N}{\mathbin}{AMSb}{"4E}
\DeclareMathSymbol{\Z}{\mathbin}{AMSb}{"5A}
\DeclareMathSymbol{\R}{\mathbin}{AMSb}{"52}
\DeclareMathSymbol{\Q}{\mathbin}{AMSb}{"51}
\DeclareMathSymbol{\I}{\mathbin}{AMSb}{"49}
\DeclareMathSymbol{\C}{\mathbin}{AMSb}{"43}

%\setstretch{1.5}

\renewcommand{\baselinestretch}{1.5}

\begin{document}

\textbf{Name: Taylor Heilman}    \hspace{4in} \textbf{Due: Fri. Dec. 7}
\begin{center} \textbf{Math 210: Homework 6 - Fall 2015} \end{center}

\begin{enumerate}

\item 

{PF

We proceed by Induction

Since $a_{3} = 2 + 1 = a_{2} + a_{1}$ it is true for $n = 3$

Assume the equation is true for an arbitrary $k$, where $k \geq 4$

So $a_{k} = a_{k-1} + a_{k-2}$

Observe that $a_{k+1} = 2a_{k} + a_{k-2}$

We know that $ a_{k-2} = a_{k} - a_{k-1}$

So $a_{k+1} = 2a_{k} - (a_{k} - a_{k-1})$

$\Rightarrow  2a_{k} - a_{k} + a_{k-1})$

$\Rightarrow  a_{k}  + a_{k-1})$

Hence $a_{k+1} $

By the Principal of Mathematical Induction the formula is true for every integer $n \geq 3$
}


\item 

{PF

a.)


b.)


//
}





\item 

{PF

We proceed by contradiction.

Assume that the sum of $\sqrt{2} + \sqrt{5} $is rational, or

$\sqrt{2} + \sqrt{5} = \frac{a}{b}$ where $a,b \in \Z$ and $b \neq 0$

$\Rightarrow \sqrt{5} = \frac{a}{b} - \sqrt{2}$

$\Rightarrow 5 = \frac{a^2}{b^2} - 2 \sqrt{2}\frac{a}{b} + 2$

$\Rightarrow \sqrt{2} = \frac{-3b}{2a} +  \frac{a}{2b}$

This equation implies that $\sqrt{2}$ is a rational number, which causes a contradiction

Therefore $\sqrt{2} + \sqrt{5} $ is and irrational number
//
}

\item 

{PF


//
}

\item 

{PF

a.)

Let $x,y \in A$

Assume that $f(x) = f(y)$

Hence we have $x = i_{a}(x)$

= $(g$ o $f)(x)$

= $g(f(x))$

= $g(f(y))$

= $(g $ o $f)(y)$

= $i_{a}(y)$

= $y$

Therefore $f$ is one-to-one

Now let $a \in A$ and set $b= f(a)$

Then $g(b) = g(f(a))$

= $(g$ o $f)(a)$

= $i_{A}(a)$

= $a$

Therefore $g$ is onto


b.)

//
}









\end{enumerate}

\end{document}