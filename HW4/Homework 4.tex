\documentclass[12pt]{article}
\usepackage{amsmath}


\pagestyle{headings} \thispagestyle{empty}
%\pagestyle{empty}
  \textwidth      6.4in
      \oddsidemargin 0.0in
      \topmargin     -0.4in
      \topskip          0pt
      \headheight      12pt
      \footskip        18pt
%      \footheight      12pt
      \textheight     650pt

\parindent=0cm
\baselineskip=2cm

%\include these lines if you want to use the LaTeX "theorem" environments
\newtheorem{theorem}{Theorem}[section]
\newtheorem{definition}[theorem]{Definition}
\newtheorem{lemma}[theorem]{Lemma}
\newtheorem{corollary}[theorem]{Corollary}
\newtheorem{guess}{Conjecture}
\newtheorem{example}[theorem]{Example}

%include lines like this if you want to define your own commands
%to save typing
\newcommand{\PROOF}{\noindent {\bf Proof}: }
\newcommand{\REF}[1]{[\ref{#1}]}
\newcommand{\Ref}[1]{(\ref{#1})}
\newcommand{\dt}{\mbox{\rm   dt}}
\newcommand{\qed}{\Large{\bf{$\diamond$}}}
\newcommand{\phat}{\hat{p}}

\DeclareSymbolFont{AMSb}{U}{msb}{m}{n}
\DeclareMathSymbol{\N}{\mathbin}{AMSb}{"4E}
\DeclareMathSymbol{\Z}{\mathbin}{AMSb}{"5A}
\DeclareMathSymbol{\R}{\mathbin}{AMSb}{"52}
\DeclareMathSymbol{\Q}{\mathbin}{AMSb}{"51}
\DeclareMathSymbol{\I}{\mathbin}{AMSb}{"49}
\DeclareMathSymbol{\C}{\mathbin}{AMSb}{"43}

%\setstretch{1.5}

\renewcommand{\baselinestretch}{1.5}

\begin{document}

\textbf{Name: Taylor Heilman}    \hspace{4in} \textbf{Due: Wed. Oct 7}
\begin{center} \textbf{Math 210: Homework 4 - Spring 2015} \end{center}

\begin{enumerate}

\item (Problem 4.36)  Prove that for every positive real number $x$ that $1+ \frac{1}{x^4} \geq \frac{1}{x} + \frac{1}{x^3}$.

{PF

${x^4} (1+ \frac{1}{x^4} \geq \frac{1}{x} + \frac{1}{x^3})$

$\Rightarrow$  ${x^4} + 1 \geq {x^3} + x$

$\Rightarrow$ ${x^4} + 1 - {x^3} - x \geq 0$

$\Rightarrow$ ${x^3} (x + \frac{1}{x^3})-({x^3} - x)   \geq 0$

$\Rightarrow$ $({x^3}-1)({x-1})  \geq 0$

Since  $({x^3}-1)({x - 1})  \geq 0$, it follows that ${x^4} + 1 - {x^3} - x \geq 0$ and so ${x^4} + 1 - {x^3} - x \geq 0$

Dividing this by  ${x^4}$ we get the desired  result.

 $1+ \frac{1}{x^4} \geq \frac{1}{x} + \frac{1}{x^3}$.





//
}
\item (Problem 4.46) Let $A$ and $B$ be sets.  Prove that $A \cup B = A \cap B$ if and only if $A = B$.

{PF

$(\Rightarrow)$ By Contrapositive

Assume $A \neq  B$

Let $a \in A$ but $a \notin B$

Then $a \in A \cup B$

and $a \notin A \cap B$

So  $A \cup B \neq A \cap B$ 

$(\Leftarrow)$

Given A = B

So for every $a \in A$ then there is also $a \in B$

Then $A \cup B$ = $A$ = $B$

and $A \cap B$ = $A$ = $B$

Therefore $A \cup B = A \cap B$




//
}



\item (Problem 4.56) Let $A, B$ and $C$ be sets.  Prove that $(A - B) \cup (A - C) = A - (B \cap C)$.
{PF

$(\subseteq )$

Let $x \in (A - B) \cup (A - C)$

wlog Assume $x \in A \times B$

By definition $x \in A$ and $x \notin B$ 

Similarly $x \in A$ and $x \notin C$ 

Therefore $x \in (A - B) \cup (A - C)$

$(\supseteq)$

Let $x \in A - ( B \cap  C)$

By definition $x \notin B \cap C$ 

Then $x \in A$ and either $x \notin B$ and  $x \in C$, or $x \in B$ and  $x \notin C$, or $x \notin B$ and  $x \notin C$
\vspace{5mm}

\textbf{Case 1:}  $x \notin B$ and  $x \in C$

Since $x \notin B$

By definition $x \notin B \cap C$

\vspace{5mm}

\textbf{Case 2:}  $x \in B$ and  $x \notin C$

Since $x \notin C$

By definition $x \notin B \cap C$

\vspace{5mm}


\textbf{Case 3:}  $x \notin B$ and  $x \notin C$

Since $x \notin C$ and  $x \notin C$

By definition $x \notin B \cap C$

\vspace{5mm}

Therefore $x \in A - ( B \cup  C)$

So $(A - B) \cup (A - C)$ = $A - ( B \cup  C)$

//
}




\item (Problem 4.58) Let $A, B$ and $C$ be sets.  Prove that $A \cap \overline{(B \cap \overline{C})} = \overline{(\overline{A} \cup B) \cap (\overline{A} \cup \overline{C})}$.

{PF

$A \cap \overline{(B \cap \overline{C})}$ 

$=$  $A \cap {(\overline B \cup  \overline { \overline C})}$

 $=$ $(A \cap \overline B) \cup (A \cap \overline {\overline C})$

$=$ $\overline {(A \cup \overline B)} \cap (\overline{A \cup \overline {\overline C})}$

$=$ $ \overline{(\overline{A} \cup B) \cap (\overline{A} \cup \overline{C})}$.

//
}

\item (Problem 4.68) Let $A, B, C$ and $D$ be sets.  Prove that \newline $(A \times B) \cap (C \times D) = (A \cap C) \times (B \cap D)$.

{PF

	$(\subseteq)$

	Let $(x,y) \in (A \times B) \cap (B \times D)$
	
	Then $(x,y) \in A \times B $ and $ (x,y) \in B \times D$
	
	Thus $(x,y) \in A \times B$ and $(x,y) \in  C \times D$
	
	By definition $(x,y) \in (A \times B) \cap (C \times D)$
	
	
	$(\supseteq)$
	
	Let $(x,y) \in ( A \cap C) \times (B \cap D)$
	
	thus $x \in (A \cap C$) and $y \in (B \cap D)$
	
	Then  $x \in A$, $x \in C$ 
	
	Also  $y \in B$ and $y \in D$
	
	By definition $(x,y) \in ( A \cap C) \times (B \cap D)$
	
	So $(A \times B) \cap (B \times D)$ = $( A \cap C) \times (B \cap D)$
	
	//


}



\end{enumerate}

\end{document}