\documentclass[12pt]{article}
\usepackage{amsmath}


\pagestyle{headings} \thispagestyle{empty}
%\pagestyle{empty}
  \textwidth      6.4in
      \oddsidemargin 0.0in
      \topmargin     -0.4in
      \topskip          0pt
      \headheight      12pt
      \footskip        18pt
%      \footheight      12pt
      \textheight     650pt

\parindent=0cm
\baselineskip=2cm

%\include these lines if you want to use the LaTeX "theorem" environments
\newtheorem{theorem}{Theorem}[section]
\newtheorem{definition}[theorem]{Definition}
\newtheorem{lemma}[theorem]{Lemma}
\newtheorem{corollary}[theorem]{Corollary}
\newtheorem{guess}{Conjecture}
\newtheorem{example}[theorem]{Example}

%include lines like this if you want to define your own commands
%to save typing
\newcommand{\PROOF}{\noindent {\bf Proof}: }
\newcommand{\REF}[1]{[\ref{#1}]}
\newcommand{\Ref}[1]{(\ref{#1})}
\newcommand{\dt}{\mbox{\rm   dt}}
\newcommand{\qed}{\Large{\bf{$\diamond$}}}
\newcommand{\phat}{\hat{p}}

\DeclareSymbolFont{AMSb}{U}{msb}{m}{n}
\DeclareMathSymbol{\N}{\mathbin}{AMSb}{"4E}
\DeclareMathSymbol{\Z}{\mathbin}{AMSb}{"5A}
\DeclareMathSymbol{\R}{\mathbin}{AMSb}{"52}
\DeclareMathSymbol{\Q}{\mathbin}{AMSb}{"51}
\DeclareMathSymbol{\I}{\mathbin}{AMSb}{"49}
\DeclareMathSymbol{\C}{\mathbin}{AMSb}{"43}

%\setstretch{1.5}

\renewcommand{\baselinestretch}{1.5}

\begin{document}

\textbf{Name:}    \hspace{4in} \textbf{Due: Fri. Nov. 6}
\begin{center} \textbf{Math 210: Homework 6 - Fall 2015} \end{center}

\begin{enumerate}

\item Let $a_1 = 2, a_2 = 4,$ and $a_{n+2} = 5a_{n+1} - 6a_n$ for all $n \geq 1$.  Prove that $a_n = 2^n$ for all natural numbers $n \geq 1$.
{PF

We proceed by Induction.

Assume n = 1

So, $a_1 = 2 = 2^1$

Assume that $a_{k+2} = 5a_{k+1}- 6a_k$ for all $k \geq 1$ and  $a_k = 2^k$

Let $k = n + 1$

$a_{k+2} = 5a_{k+1}- 6a_k$

which equals  $(5)2^{k+2}  - (6)2^{k+1}$

= $(5)(2^2)(2^k) - (6)(2^k)(2)$

= $20(2^k) - 12(2^k)$

= $8(2^k)$

= $2^3(2^k)$

= $2^k+3$

//
}





\item Use induction to show that for all $n \in \N$,
$$ \sum_{i=1}^{n} \frac{1}{(2i - 1)(2i + 1)} = \frac{n}{2n + 1}.$$
{PF

Let $n = 1$, $ \sum_{i=1}^{1} \frac{1}{(2(1) - 1)(2(1) + 1)} = \frac{1}{3}  = \frac{n}{2n + 1}.$

Therefore $n = 1$ is true.

Suppose $ \sum_{i=1}^{k} \frac{1}{(2i - 1)(2i + 1)} = \frac{k}{2k + 1}$ for all $\N k \geq 1$

Consider  $ \sum_{i=1}^{k+1} \frac{1}{(2i - 1)(2i + 1)}$

=   $ \sum_{i=1}^{k} \frac{1}{(2i - 1)(2i + 1)}+  \frac{1}{(2(k+1) - 1)(2(k+1) + 1)}$

= $\frac{k}{(2k + 1)}+  \frac{1}{(2k+2-1) (2k+2+ 1)}$

= $\frac{k}{(2k + 1)}+  \frac{1}{(2k+1)(2k+3)}$

= $\frac{(2k+1)(k+1)}{(2k + 1)(2k+3)}$

= $\frac{(k+1)}{(2k+3)}$

= $\frac{(k+1)}{(2(k+1)+1)}$

The result then follows by the Principle of Induction.

//

}





\item (Problem 6.44) Consider the sequence $f_0, f_1, f_2, \hdots$ where 
$$f_0 = 1, f_1 = 1, f_2 = 2, f_3 = 3, f_4 = 5 \text{ and } f_5 = 8.$$
The terms in this sequence are called \emph{Fibonacci Numbers}.
\begin{enumerate}
\item Define the sequence of Fibonacci numbers by means of a recurrence relation.
\item Prove that $2 \mid f_n$ if and only if $3 \mid n$.

{PF

a.)  $F1 = F2 = 1$ and $Fn = Fn -1 + Fn - 2$ for $n \leq 3$

b.) We proceed by Induction. 

For n=1, $f3n-2 , f2n-1 , f3n$ = $f1, f2, f3$ = 1, 1, 2

Notice f1 = f2 = 1, which is odd, while f3 = 2 which is even.

Therefore the it is true for n = 1.

Let $k \geq 1$, for some $k \in \N$, where $f3k-2 , f3k-1 , f3k$ = ($2l+1, 2m+1, 2n$), where $l,m,n$ are $\in \Z$

Observe that for $f_n$ we have $f3k+1= f3k + f3k-1$ 

So $f3k+1$ is the sum of $f3k$, an even number, and $f3k-1$, an odd number.

Therefore $f3k+1$ is odd

Similarly $f3k+2$ is the sum of $f3k$, an even number, and $f3k+1$, an odd number.

Therefore $f3k+2$ is odd

Lastly, $F3k+3 = F3k+2 + F3k+1$

So $F3k+3$ is the sum of two odd integers and is therefore even.

The result then follows by the Principle of Induction.

//
}



\end{enumerate}

\item Use induction to show that the Fibonacci numbers satisfy the formula
$$f_n = \frac{1}{\sqrt{5}} \left( \frac{1 + \sqrt{5}}{2} \right)^n - \frac{1}{\sqrt{5}} \left( \frac{1 - \sqrt{5}}{2} \right)^n, n \geq 0.$$

{PF

For $f_n = \frac{1}{\sqrt{5}} \left( \frac{1 + \sqrt{5}}{2} \right)^n - \frac{1}{\sqrt{5}} \left( \frac{1 - \sqrt{5}}{2} \right)^n, n \geq 0.$ let $n = 0$

Then $f_1 = \frac{1}{\sqrt{5}}(\frac{1+\sqrt{5}}{2})^{2} - \frac{1}{\sqrt{5}}(\frac{1-\sqrt{5}}{2})^{2}$

So $f_1 = 1$

Let $f_n = f_{n-2} + f_{n-1}$
and $f_{n+1} = f_{n-1} + f_n$

Then $f_{k+1} = f_{k-1} + f_{k} = \frac{1}{\sqrt{5}}$[($\frac{1+\sqrt{5}}{2})^{k+2}-(\frac{1-\sqrt{5}}{2})^{k+2}] $

So $f_{k}  = \frac{1}{\sqrt{5}}(\frac{1+\sqrt{5}}{2})^{k+1} - \frac{1}{\sqrt{5}}(\frac{1-\sqrt{5}}{2})^{k+1} $

And  $f_{k-1}  = \frac{1}{\sqrt{5}}(\frac{1+\sqrt{5}}{2})^{k} - \frac{1}{\sqrt{5}}(\frac{1-\sqrt{5}}{2})^{k} $

Then $f_{k+1} = \frac{1}{\sqrt{5}}[((\frac{1+\sqrt{5}}{2})^{k+1} - \frac{1}{\sqrt{5}}(\frac{1-\sqrt{5}}{2})^{k+1}] + \frac{1}{\sqrt{5}}[(\frac{1+\sqrt{5}}{2})^{k} - \frac{1}{\sqrt{5}}(\frac{1-\sqrt{5}}{2})^{k}] $

= $\frac{1}{\sqrt{5}}[(\frac{1+\sqrt{5}}{2})^{k+2}-(\frac{1-\sqrt{5}}{2})^{k+2}]$

the result  follows the Principle of Mathematical Induction.

//
}

\item Use induction to show that for all $n \in \N$, the Fibonacci numbers satisfy:
$$f_0 + f_1 + \hdots + f_{n} = f_{n+2} - 1$$ 
{PF

Let $f_i = f_{n+2} - 1$

Then when $n =1$, $f_3 -1 = 2 - 1 = 1$

So $f_i = f_1 = 1$ 

Hence, $n = 1$ is true

Now let $k \in \N$ and suppose True for $n = k$


$\sum\limits_{i=1}^{k+1} f_i = \sum\limits_{i=1}^{k} f_i + f_{k+1}$


Where $f_i = f_{k+2} - 1 $

So $f_i = ( f_{k+2} - 1 )+ f_{k+1} $

= $ f_{k+3} - 1$

This holds true for $n = k +1$ so the result  follows the Principle of Mathematical Induction.

//
}









\end{enumerate}

\end{document}