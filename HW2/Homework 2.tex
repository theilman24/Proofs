\documentclass[12pt]{article}


\pagestyle{headings} \thispagestyle{empty}
%\pagestyle{empty}
  \textwidth      6.4in
      \oddsidemargin 0.0in
      \topmargin     -0.4in
      \topskip          0pt
      \headheight      12pt
      \footskip        18pt
%      \footheight      12pt
      \textheight     650pt

\parindent=0cm
\baselineskip=2cm

%\include these lines if you want to use the LaTeX "theorem" environments
\newtheorem{theorem}{Theorem}[section]
\newtheorem{definition}[theorem]{Definition}
\newtheorem{lemma}[theorem]{Lemma}
\newtheorem{corollary}[theorem]{Corollary}
\newtheorem{guess}{Conjecture}
\newtheorem{example}[theorem]{Example}

%include lines like this if you want to define your own commands
%to save typing
\newcommand{\PROOF}{\noindent {\bf Proof}: }
\newcommand{\REF}[1]{[\ref{#1}]}
\newcommand{\Ref}[1]{(\ref{#1})}
\newcommand{\dt}{\mbox{\rm   dt}}
\newcommand{\qed}{\Large{\bf{$\diamond$}}}
\newcommand{\phat}{\hat{p}}

\DeclareSymbolFont{AMSb}{U}{msb}{m}{n}
\DeclareMathSymbol{\N}{\mathbin}{AMSb}{"4E}
\DeclareMathSymbol{\Z}{\mathbin}{AMSb}{"5A}
\DeclareMathSymbol{\R}{\mathbin}{AMSb}{"52}
\DeclareMathSymbol{\Q}{\mathbin}{AMSb}{"51}
\DeclareMathSymbol{\I}{\mathbin}{AMSb}{"49}
\DeclareMathSymbol{\C}{\mathbin}{AMSb}{"43}

%\setstretch{1.5}

\renewcommand{\baselinestretch}{1.5}

\begin{document}

\textbf{Name: Taylor Heilman}    \hspace{4in} \textbf{Due: Friday, Sept 18}
\begin{center} \textbf{Math 210: Homework 2 - Spring 2015} \end{center}

\begin{enumerate}

\item Problem 1.8: Let $A = \{ n \in \Z \mid 2 \leq |n| < 4 \}, B = \{ x \in \Q \mid 2 < x \leq 4 \}, \newline C = \{ x \in \R \mid x^2-(2 + \sqrt{2})x + 2\sqrt{2} = 0 \}$ and $D = \{ x \in \Q \mid x^2 - (2 + 2\sqrt{2})x + 2\sqrt{2} = 0 \}$
\begin{enumerate}
\item Describe $A$ by listing its elements.

A = \{-3,-2,2,3\}
\item Give an example of three elements that belong to $B$ but do not belong to $A$.

\{5/2, 7/3, 8/3\}
\item Describe the set $C$ by listing its elements.

$C = \{ \sqrt{2}, 2\}$
\item Describe the set $D$ in another manner.

$D = \{ x \in \Q \mid -( \sqrt 2 - x)(x-2)=0\}$

\item Determine the cardinality of each of the sets $A$, $B$, $C$, and $D$.

$ |A| = 2, |B| = \infty , |C| = 2, |D| = 0 $


\end{enumerate}

\item Problem 1.18: For $A = \{ x \mid x = 0$ or $x \in \mathcal{P}(\{ 0 \}) \}$, determine $\mathcal{P}(A)$.

$\mathcal{P}(A) = \{ \emptyset , 0, \{0\} , \{\{0\}\}  \}$


\item Problem 1.30: Let $A = \{ x \in R \mid | x-1 | \leq 2 \}, B = \{ x \in R \mid |x| \geq 1 \}$, and $C = \{ x \in \R \mid |x+2| \leq 3 \}$.
\begin{enumerate}
\item Express $A, B$, and $C$ using interval notation.

$A = [-1,3]$
$B = (- \infty ,-1) U (1, \infty) $
$C = [-5,1]$
\item Determine each of the following sets using interval notation: $A \cup B, A \cap B, \newline B \cap C, B - C$.


$A \cup B = (- \infty , \infty )$ \newline
$A \cap B = (1,3]  $ \newline
$B \cap C = [-5, -2]$ \newline
$B - C = (- \infty ,-5)U (1, \infty )$

\end{enumerate}

\item For $i \in \Z$, let $A_i = \{ i -1, i+1 \}$.  Determine the following:
\begin{enumerate}
\item $ \displaystyle \bigcup_{i=1}^{5} A_{2i}$


$A_2 U A_4 U A_6 U A_8 U A_10 $ 

$A = \{ 1,3,5,7,9,11\}$
\item $\displaystyle \bigcup_{i=1}^{5} (A_i \cap A_{i+1})$

$(A_1 U A_2 U A_3 U A_4 U A_5) U (A_2 U A_3 U A_4 U A_5 U A_6)$


$(A_i \cap A_{i+1}) = \{ 1,2,3,4,5,6\}$

\item $\displaystyle \bigcup_{i=1}^{5} (A_{2i-1} \cap A_{2i+1})$

$(A_1 U A_3 U A_5 U A_7 U A_9) U (A_3 U A_5 U A_7 U A_9 U A_11)$

$(A_{2i-1} \cap A_{2i+1}) = \{2,4,6,8,10 \}$

\end{enumerate}

\item For $A = \{ a \in \R \mid |a| \leq 1 \}$ and $B = \{ b \in \R \mid |b| = 1 \}$, give a geometric description of the points in the $xy$-plane belonging to $(A \times B) \cup (B \times A)$.

\textbf{The points in the set $(A \times B) $ will create two parallel, horizontal lines intersecting the Y-axis at 1 and -1. The points in the set $(B \times A ) $ will create two parallel, vertical lines that intersect the X-axis at 1 and -1.  The union of the two will create 4 points at (-1,-1), (-1,1),(1,-1),(1,1)}

\end{enumerate}

\end{document}